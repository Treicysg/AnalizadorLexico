\documentclass[10, usernames, dvipsnames]{beamer}
\usepackage{color}
\usepackage[spanish]{babel}
\setbeamercolor{itemize item}{fg=darkred!80!black}
\begin{document} 
\title{
\begin{LARGE}
Instituto Tecnol\'ogico de Costa Rica
\end{LARGE}
\newline
\begin{Large}
\\Compiladores e Int\'erpretes
\\Proyecto \#2: Analizador L\'exico
\\Profesor: Francisco Torres
\end{Large}
}
\author{Dennisse Rojas Casanova
\\Treicy S\'anchez Guti\'errez}
\date{25 de Mayo, 2016}
\maketitle 
\begin{frame} 
\frametitle{An\'alisis L\'exico y Flex} 
El An\'alisis L\'exico consiste en descomponer un fuente de entrada en categor\'ias l\'exicas m\'inimas llamadas tokens.Un programa en Flex consiste b\'asicamente en una lista de expresiones regulares que definen acciones a ejecutar cuando ocurre un match.\end{frame} 
\begin{frame}
\textcolor{BurntOrange}{6}
\end{frame}
\end{document}