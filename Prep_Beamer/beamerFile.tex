\documentclass[8, usernames, dvipsnames]{beamer}
\usepackage{color}
\usepackage{pgfplots}
\pgfplotsset{width=11.5cm,compat=1.9}
\usepackage{dirtytalk}
\usepackage[spanish]{babel}
\usepackage[utf8]{inputenc}
\setbeamercolor{itemize item}{fg=darkred!80!black}
\begin{document} 
\title{
\begin{LARGE}
Instituto Tecnol\'ogico de Costa Rica
\end{LARGE}
\newline
\begin{Large}
\\Compiladores e Int\'erpretes
\\Proyecto \#2: Analizador L\'exico
\\Profesor: Francisco Torres
\end{Large}
}
\author{Dennisse Rojas Casanova
\\Treicy S\'anchez Guti\'errez}
\date{25 de Mayo, 2016}
\maketitle 
\begin{frame} 
\frametitle{An\'alisis L\'exico y Flex} 
El An\'alisis L\'exico consiste en descomponer un fuente de entrada en categor\'ias l\'exicas m\'inimas llamadas tokens.Un programa en Flex consiste b\'asicamente en una lista de expresiones regulares que definen acciones a ejecutar cuando ocurre un match.\end{frame} 
\begin{frame}
\textcolor{OrangeRed}{void}
\textcolor{White}{\ }
\textcolor{Violet}{open}\textcolor{Sepia}{\_}
\textcolor{Violet}{file}\textcolor{SkyBlue}{(}
\textcolor{SkyBlue}{)}
\textcolor{SkyBlue}{\{ }

 \textcolor{White}{\ }
\textcolor{White}{\ }
\textcolor{White}{\ }
\textcolor{OrangeRed}{char}
\textcolor{White}{\ }
\textcolor{Violet}{filename}\textcolor{White}{\ }
\textcolor{SkyBlue}{[}
\textcolor{SeaGreen}{100}
\textcolor{SkyBlue}{]}
\textcolor{Salmon}{=}
\textcolor{Sepia}{;}

 \textcolor{White}{\ }
\textcolor{White}{\ }
\textcolor{White}{\ }
\textcolor{Violet}{printf}\textcolor{SkyBlue}{(}
\textcolor{White}{\ }
\textcolor{Violet}{Enter}\textcolor{White}{\ }
\textcolor{Violet}{a}\textcolor{White}{\ }
\textcolor{Violet}{value}\textcolor{White}{\ }
\textcolor{Sepia}{:}
\textcolor{White}{\ }
\textcolor{SkyBlue}{)}
\textcolor{Sepia}{;}

 \textcolor{White}{\ }
\textcolor{White}{\ }
\textcolor{White}{\ }
\textcolor{Violet}{scanf}\textcolor{SkyBlue}{(}
\textcolor{Violet}{s}\textcolor{Sepia}{,}
\textcolor{Violet}{filename}\textcolor{SkyBlue}{)}
\textcolor{Sepia}{;}

 \textcolor{White}{\ }
\textcolor{White}{\ }
\textcolor{White}{\ }
\textcolor{Violet}{file}\textcolor{White}{\ }
\textcolor{Salmon}{=}
\textcolor{White}{\ }
\textcolor{Violet}{fopen}\textcolor{SkyBlue}{(}
\textcolor{White}{\ }
\textcolor{Violet}{filename}\textcolor{Sepia}{,}
\textcolor{White}{\ }
\textcolor{Violet}{r}\textcolor{White}{\ }
\textcolor{SkyBlue}{)}
\textcolor{Sepia}{;}

 \textcolor{SkyBlue}{\} }

 \end{frame}
\begin{frame} 
\frametitle{Histograma} 
\begin{tikzpicture} 
\begin{axis}[ 
symbolic x coords={Reservadas, Identificadores,Literales,Operadores, Delimitadores},xtick=data
]
\addplot[ybar,fill=Turquoise]  
coordinates {
(Reservadas,2)(Identificadores,14)(Literales,1)(Operadores,3)(Delimitadores,19)};
\end{axis}
\end{tikzpicture}
\end{frame} 
\end{document}