\documentclass[10, usernames, dvipsnames]{beamer}
\usepackage{color}
\usepackage{pgfplots}
\pgfplotsset{width=10cm,compat=1.9}
\usepackage{dirtytalk}
\usepackage[spanish]{babel}
\usepackage[utf8]{inputenc}
\setbeamercolor{itemize item}{fg=darkred!80!black}
\begin{document} 
\title{
\begin{LARGE}
Instituto Tecnol\'ogico de Costa Rica
\end{LARGE}
\newline
\begin{Large}
\\Compiladores e Int\'erpretes
\\Proyecto \#2: Analizador L\'exico
\\Profesor: Francisco Torres
\end{Large}
}
\author{Dennisse Rojas Casanova
\\Treicy S\'anchez Guti\'errez}
\date{25 de Mayo, 2016}
\maketitle 
\begin{frame} 
\frametitle{An\'alisis L\'exico y Flex} 
El An\'alisis L\'exico consiste en descomponer un fuente de entrada en categor\'ias l\'exicas m\'inimas llamadas tokens.Un programa en Flex consiste b\'asicamente en una lista de expresiones regulares que definen acciones a ejecutar cuando ocurre un match.\end{frame} 
\begin{frame}
\textcolor{OrangeRed}{int}
\textcolor{White}{\ }
\textcolor{Violet}{main}\textcolor{SkyBlue}{(}
\textcolor{SkyBlue}{)}

 \textcolor{SkyBlue}{\{ }

 \textcolor{White}{\ }
\textcolor{White}{\ }
\textcolor{OrangeRed}{if}
\textcolor{SkyBlue}{(}
\textcolor{RedViolet}{2}
\textcolor{White}{\ }
\textcolor{SpringGreen}{+}
\textcolor{White}{\ }
\textcolor{RedViolet}{5}
\textcolor{White}{\ }
\textcolor{OliveGreen}{$==$}
\textcolor{White}{\ }
\textcolor{RedViolet}{7}
\textcolor{SkyBlue}{)}
\textcolor{SkyBlue}{\{ }

 \textcolor{White}{\ }
\textcolor{White}{\ }
\textcolor{White}{\ }
\textcolor{White}{\ }
\textcolor{Violet}{printf}\textcolor{SkyBlue}{(}
\textcolor{Orchid}{\say{Hola}}
\textcolor{SkyBlue}{)}
\textcolor{Sepia}{;}

 \textcolor{White}{\ }
\textcolor{White}{\ }
\textcolor{SkyBlue}{\} }

 \textcolor{SkyBlue}{\} }

 \end{frame}
\begin{frame} 
\frametitle{Histograma} 
\begin{tikzpicture} 
\begin{axis}[ 
x tick label style={
/pgf/number format/1000 sep=},
ylabel=Cantidad de Tokens,
enlargelimits=0.05,
legend style={at={(0.5,-0.1)},
anchor=north,legend columns=-1},
ybar interval=0.7,
]
\addplot 
coordinates {(49,1)(33,13)(12,2)(5,3)(6,3)};
\end{axis}
\end{tikzpicture}
\end{frame}
\begin{frame} 
\begin{tikzpicture} 
\begin{axis}[ 
x tick label style={
/pgf/number format/1000 sep=},
ylabel=Cantidad de Tokens,
enlargelimits=0.05,
legend style={at={(0.5,-0.1)},
anchor=north,legend columns=-1},
ybar interval=0.7,
]
\addplot 
coordinates {(34,6)(9,2)(3,1)(13,3)(7,2)};
\end{axis}
\end{tikzpicture}
\end{frame}
\begin{frame} 
\begin{tikzpicture} 
\begin{axis}[ 
x tick label style={
/pgf/number format/1000 sep=},
ylabel=Cantidad de Tokens,
enlargelimits=0.05,
legend style={at={(0.5,-0.1)},
anchor=north,legend columns=-1},
ybar interval=0.7,
]
\addplot 
coordinates {(14,2)(11,2)(10,3)(-1,2)};
\end{axis}
\end{tikzpicture}
\end{frame} 
\end{document}